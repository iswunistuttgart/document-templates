% \iffalse meta-comment
%
% Copyright (C) 2019--2020 by Philipp Tempel <latex@philipptempel.me>
% -------------------------------------------------------
% 
% This file may be distributed and/or modified under the
% conditions of the LaTeX Project Public License, either version 1.3
% of this license or (at your option) any later version.
% The latest version of this license is in:
%
%    http://www.latex-project.org/lppl.txt
%
% and version 1.3 or later is part of all distributions of LaTeX 
% version 2005/12/01 or later.
%
% \fi
%
% \iffalse
%<*driver>
\ProvidesFile{ustuttdemos.dtx}[2020/02/14 v1.2.0 University of Stuttgart LaTeX demo files generator]
%</driver>
%
%<*driver>
\documentclass{ltxdoc}
\EnableCrossrefs         
\CodelineIndex
\RecordChanges
\begin{document}
  \DocInput{ustuttdemos.dtx}
  \PrintChanges
  \PrintIndex
\end{document}
%</driver>
% \fi
%
% \CheckSum{0}
%
% \CharacterTable
%  {Upper-case    \A\B\C\D\E\F\G\H\I\J\K\L\M\N\O\P\Q\R\S\T\U\V\W\X\Y\Z
%   Lower-case    \a\b\c\d\e\f\g\h\i\j\k\l\m\n\o\p\q\r\s\t\u\v\w\x\y\z
%   Digits        \0\1\2\3\4\5\6\7\8\9
%   Exclamation   \!     Double quote  \"     Hash (number) \#
%   Dollar        \$     Percent       \%     Ampersand     \&
%   Acute accent  \'     Left paren    \(     Right paren   \)
%   Asterisk      \*     Plus          \+     Comma         \,
%   Minus         \-     Point         \.     Solidus       \/
%   Colon         \:     Semicolon     \;     Less than     \<
%   Equals        \=     Greater than  \>     Question mark \?
%   Commercial at \@     Left bracket  \[     Backslash     \\
%   Right bracket \]     Circumflex    \^     Underscore    \_
%   Grave accent  \`     Left brace    \{     Vertical bar  \|
%   Right brace   \}     Tilde         \~}
%
%
% \changes{v1.0}{2019/12/07}{Initial version}
%
% \GetFileInfo{ustuttdemos.dtx}
%
% \DoNotIndex{\",\@,\abs,\acronymtype,\and,\begin,\bigg,\bm,\cdoot,\dof,\eg,\ell,\end,\ensuremath,\equiv,\eye,\fixcleverefappendix,\geq,\gram,\ie,\in,\kilo,\Maass,\mass,\mathds,\mc,\ms,\nicefrac,\norm,\numposconstraints,\parentheses,\pow,\printdate,\secnd,\si,\sum,\textemdash,\times,\today,\transpose,\usepackage,\vect}
% 
%
% \title{The \textsf{ustuttdemos} package\thanks{This document
%   corresponds to \textsf{ustuttdemos}~\fileversion, dated \filedate.}}
% \author{Philipp Tempel \\ \texttt{latex@philipptempel.me}}
%
% \maketitle
%
% \section{Introduction}
%
% Put text here.
%
% \section{Usage}
%
% Put text here.
%
% \StopEventually{}
%
% \section{Implementation}
%
% \subsection{Documents}
%
%<*document>
%
% \begin{macro}{\documentclass}
% First off, we need to define the correct document class.
%    \begin{macrocode}
\documentclass[
%    \end{macrocode}
% The first few lines of our document class will globally determine which
% languages we want to have loaded. These global options will then apply not
% only to the corresponding document class, but also other packages like |babel|
% or |isodate|.
%
% Load German as main language
%    \begin{macrocode}
%<*german>
    english,
    ngerman,
%</german>
%    \end{macrocode}
% Load English as main language
%    \begin{macrocode}
%<*english>
  ngerman,
  english,
%</english>
%    \end{macrocode}
% 
% Now we will define the main thesis type.
%    \begin{macrocode}
%<bachelor>    degree=bachelor,
%<master>    degree=master,
%<doctorate>    degree=doctorate,
%    \end{macrocode}
% And some more, custom options can follow here
%    \begin{macrocode}
    paper=a4paper,%
    fancychapter,%
    fancyfooter,%
    % colorful,%
%    \end{macrocode}
%
% Lastly, we define the correct document class to use.
%    \begin{macrocode}
%<article>  ]{ustuttartcl}
%<book>  ]{ustuttbook}
%<thesis>  ]{ustuttthesis}
%    \end{macrocode}
% \end{macro}
%
% \begin{macro}{\usepackage}
% For theses, we also include some sample packages to allow for typesetting of
% the different glossaries
%    \begin{macrocode}
%<*thesis>
\usepackage{ustuttgloss}
\usepackage{ustuttappendix}
\usepackage{ustutttext}
\usepackage{ustuttmath}
\usepackage{ustuttmechanics}
%    \end{macrocode}
%</thesis>
% \end{macro}
%
% Load package demo content
%    \begin{macrocode}
\usepackage[math]{blindtext}
%    \end{macrocode}
% 
% \begin{macro}{\title}
% This is what your thesis' title is going to be in the end
%    \begin{macrocode}
\title{...}
%    \end{macrocode}
% \end{macro}
%
% \begin{macro}{\subtitle}
% Some thesis may additionally contain subtitles
%    \begin{macrocode}
\subtitle{...}
%    \end{macrocode}
% \end{macro}
% 
% \begin{macro}{\author}
% Obviously, the author's name. Multiple authors \emph{must} be separated by |\and|.
%    \begin{macrocode}
\author{...}
%    \end{macrocode}
% \end{macro}
% 
% \begin{macro}{\authorrunning}
% If the document is written by multiple authors and you want a running version
% of the authors' names on the page head, use this command.
%    \begin{macrocode}
\authorrunning{...}
%    \end{macrocode}
% \end{macro}
% 
% \begin{macro}{\placeofbirth}
% Place of birth of the autor. Only applies to doctoral theses.
%    \begin{macrocode}
%<doctorate>\placeofbirth{Stuttgart}
%    \end{macrocode}
% \end{macro}
%
% \begin{macro}{\date}
% Date of your document's creation which is baked into PDF document properties,
% as well as used in the title page
%    \begin{macrocode}
\date{\today}
%    \end{macrocode}
% or if you desire a specific print date
%    \begin{macrocode}
\date{\printdate{2019-04-23}}
%    \end{macrocode}
% \end{macro}
%
% \begin{macro}{\university} 
% Name of the university in document's main language
%    \begin{macrocode}
%<german>\university{Universit\"at Stuttgart}
%<english>\university{University of Stuttgart}
%    \end{macrocode}
% \end{macro}
%
% \begin{macro}{\faculty}
% Faculty name in document's main language
%    \begin{macrocode}
%<german>\faculty{Konstruktions-, Produktions- und Fahrzeugtechnik}
%<english>\faculty{Konstruktions-, Produktions- und Fahrzeugtechnik}
%    \end{macrocode}
% \end{macro}
% 
% \begin{macro}{\department}
%    \begin{macrocode}
%<german>\department{Institut f\"ur Steuerungstechnik %
%<german>der Werkzeugmaschinen und Fertigungseinrichtungen}
%<english>\department{Institute for Control Engineering %
%<english>of Machine Tools and Manufacturing Units}
%    \end{macrocode}
% \end{macro}
%
% \begin{macro}{\departmentshort}
% Short name of the department; usually its abbreviated name
%    \begin{macrocode}
%<german>\departmentshort{ISW}
%<english>\departmentshort{ISW}
%    \end{macrocode}
% \end{macro}
% 
% \begin{macro}{\major}
% For any student thesis, put the name of your major here so it can be typeset
% correctly in the titlepage
%    \begin{macrocode}
%<*thesis>
%<bachelor|master>\major{Maschinenbau}
%</thesis>
%    \end{macrocode}
% \end{macro}
% 
% \begin{macro}{\universitylogo}
% Path to the university's logo given as a regular \LaTeX path that works for
% |\includegraphics|
%    \begin{macrocode}
%<german>\universitylogo{logo-university-de}
%<english>\universitylogo{logo-university-en}
%    \end{macrocode}
% \end{macro}
% 
% \begin{macro}{\departmentlogo}
% Similar to the university logo, path to the department's logo given as regular
% \LaTeX path that works for |\includegraphics|
%    \begin{macrocode}
%<german>\departmentlogo{logo-institute-de}
%<english>\departmentlogo{logo-institute-en}
%    \end{macrocode}
% \end{macro}
% 
% \begin{macro}{\advisor}
% \begin{macro}{\examiner}
% Name of your thesis advisor/examiner. In case of bachelor's or master's
% thesis, this is usually going to be the first and only advisor of your
% thesis. In case of doctoral theses, this is your main advisor or doctoral
% advisor. Further advisors go into the |\coadvisor| macro (for doctoral theses
% only).
%    \begin{macrocode}
%<doctorate>\advisor{PD Dr.-Ing.\@ Andreas Pott}
%<bachelor|master>\examiner{PD Dr.-Ing.\@ Andreas Pott}
%    \end{macrocode}
% \end{macro}
% \end{macro}
% 
% \begin{macro}{\coadvisor}
% \begin{macro}{\supervisor}
% Further advisors of doctoral thesis or daily supervisor(s) of bachelor 
% theses. For doctoral theses, these are the members of your doctoral graduation
% board.
%    \begin{macrocode}
%<doctorate>\coadvisor{Prof.\@ Dr.\@ Bernard Haasdonk \and Marc Gouttefarde, Ph.D.}
%<bachelor|master>\supervisor{Prof.\@ Dr.\@ Bernard Haasdonk \and Marc Gouttefarde, Ph.D.}
%    \end{macrocode}
% \end{macro}
%
% \begin{macro}{\dedication} 
% You can dedicate your thesis to anyone. Since it's a macro, do not go overboard or haywire with formatting. Keep it simple.
%    \begin{macrocode}
%<book|thesis>\dedication{This is to me!}
%    \end{macrocode}
% \end{macro}
%
% \begin{macro}{\addbibresource}
% With oour bibliography managed through |biblatex|, including the bibliography resources behaves slightly differently than with using |bibtex|.
%    \begin{macrocode}
\addbibresource{references.bib}
%    \end{macrocode}
% \end{macro}
%
% \begin{macro}{\newglossary}
% \cmd{\newglossary}\oarg{}\marg{label}\marg{log}\marg{out-ext}\marg{in-ext}\marg{caption}
% To let the |glossaries| package now that there is a new glossaries type, we
% call the |\newglossary| method with the syntax
% |\newglossary{<label>}{<log>}{<out-ext>{<in-ext>}{<caption>}|
%    \begin{macrocode}
%<thesis>\newglossary[slg]{symbol}{sls}{slo}{\translate{ustutt@glossaries@symbols}}
%<thesis>\newglossary[nlg]{notation}{nls}{nlo}{\translate{ustutt@glossaries@notation}}
%    \end{macrocode}
% \end{macro}
%
% \begin{macro}{\makeindex}
% We must trigger the |makeindex| command to enable proper indexing of macros
% and glossaries.
%    \begin{macrocode}
%<thesis>\makeindex
%    \end{macrocode}
% \end{macro}
%
% \begin{macro}{\makeglossaries}
% Likewise, glossaries cannot be typeset of they have not been made.
%    \begin{macrocode}
%<thesis>\makeglossaries
%    \end{macrocode}
% \end{macro}
%
% \begin{macro}{\loadglsentries}
% Lastly, we will load glossaries entries using the intended way. The file
% loaded is just a regular \LaTeX file, however, with proper markup for the
% glossaries definitions.
%    \begin{macrocode}
%<*thesis>
\loadglsentries{symbols}
\loadglsentries{notation}
\loadglsentries{acronyms}
%</thesis>
%    \end{macrocode}
% \end{macro}
%
% \begin{macro}{\glsaddall}
% For the sake of this demo, we will just add all defined glossaries entries
%    \begin{macrocode}
%<thesis>\glsaddall
%    \end{macrocode}
% \end{macro}
% 
% Finally, the actual document begins
% \begin{environment}{document}
%    \begin{macrocode}
\begin{document}
%    \end{macrocode}
% \end{environment}
%
% \begin{macro}{\frontmatter}
% Set up the document's front matter style: romain page numbers, et cetera.
%    \begin{macrocode}
%<book|thesis>\frontmatter
%    \end{macrocode}
% \end{macro}
%
% \begin{macro}{\maketitle}
% Create the title page
%    \begin{macrocode}
\maketitle
%    \end{macrocode}
% \end{macro}
%
%
% \begin{environment}{acknowledgements}
% What's a good thesis without acknowledging someone else's share of the result?
% Be it your (doctoral) advisor, colleagues, other students, your parents, or
% the project funding agency. Give them a warm-felt shout out.
%    \begin{macrocode}
%<*book|thesis>
\begin{acknowledgements}%
\end{acknowledgements}
%</book|thesis>
%    \end{macrocode}
% \end{environment}
%
% \begin{environment}{preface}
% A preface is an introduction to a book or other literary work written by you.
%    \begin{macrocode}
%<book|thesis>\begin{preface}
%<book|thesis>\end{preface}
%    \end{macrocode}
% \end{environment}
%
% \begin{macro}{\tableofcontents}
% Most importantly, a table of contents should be type set right after all the
% prefacing stuff.
%    \begin{macrocode}
\tableofcontents
%    \end{macrocode}
% \end{macro}
%
% \begin{macro}{\listoffigures}
% List of figures follows after the table of contents. This is most oftenly
% needed as you will have more than 4 figures.
%    \begin{macrocode}
\listoffigures
%    \end{macrocode}
% \end{macro}
%
% \begin{macro}{\listoftables}
% List of tables follows if applicable. No need to typeset a list of tables in
% case you have only some 3 or 4 tables in total.
%    \begin{macrocode}
\listoftables
%    \end{macrocode}
% \end{macro}
%
% \begin{macro}{\printglossary}
% Output list of symbols from glossaries defined above
%    \begin{macrocode}
%<*thesis>
\printglossary[%
    type=symbol,%
    style=isw-long-symbol-nogroup,%
  ]
%</thesis>
%    \end{macrocode}
%
% Also typeset glossaries on notation, if any
%    \begin{macrocode}
%<*thesis>
\printglossary[%
    type=notation,%
    style=isw-long-notation-nogroup,%
  ]
%</thesis>
%    \end{macrocode}
%
% Lastly, output our glossaries of acronyms
%    \begin{macrocode}
%<*thesis>
\printglossary[%
    type=\acronymtype,%
    style=isw-long-acronym,%
    nonumberlist,%
    nogroupskip,%
  ]
%</thesis>
%    \end{macrocode}
% \end{macro}
%
% \begin{macro}{\mainmatter}
% Switch into the main document mode resetting page numbers, changing page
% numbers to arabic, et cetera.
%    \begin{macrocode}
%<book|thesis>\mainmatter
%    \end{macrocode}
% \end{macro}
%
% Here is where you would now put your actual data content
%    \begin{macrocode}
%<<COMMENT
% Put your content here instead of the following line
%COMMENT
%<*book|thesis>
\Blinddocument
\Blinddocument
\Blinddocument
\Blinddocument
%</book|thesis>
%<*article>
\blinddocument
\blinddocument
%</article>
%    \end{macrocode}
%
% \begin{macro}{\printbibliography}
% Eventually, a list of bibligraphy items will follow
%    \begin{macrocode}
%<<COMMENT
% Remove this line in your final document so that you will only have entries in
% your bibliography that you actually cited
\nocite{*}
%COMMENT
\printbibliography
%    \end{macrocode}
% \end{macro}
%
%
% \begin{environment}{appendices}
% In case you have content that's not supposed to be in the main document part
% but still worth having it printed, it'll belong into the appendix. You enter
% |appendix|-mode by using the |appendices| environment.
%    \begin{macrocode}
% \begin{appendices}
% \fixcleverefappendix
% %<<COMMENT
% % Your appendix here, if not needed, remove the environment
% %COMMENT
% \end{appendices}
%    \end{macrocode}
% \end{environment}
%
% \begin{environment}{document}
% Finally, close the document
%    \begin{macrocode}
\end{document}
%    \end{macrocode}
% \end{environment}
%
%</document>
%
% \subsection{Mixins}
%
% \subsubsection{Symbols}
%
% Symbols can generally be typeset using the |glossaries| package, however, they
% would be added to the global glossaries list. Most often, this is not what we
% want as we want a separate list of all symbols.
%
% Now, we are able to add all our glossary entries one by one
% \begin{macro}{\newglossaryentry}
%    \begin{macrocode}
%<*symbols>
\newglossaryentry{sym:degrees-of-freedom}{%
  type=symbol,%
  name={\ensuremath{\dof}},%
  sort={degrees of freedom},%
  unit={},%
  symbol={%
    \ensuremath{\dof}%
  },%
  description={%
    Number of degrees of freedom of a rigid body.%
  },%
}
%</symbols>
%    \end{macrocode}
% \end{macro}
%
% \begin{macro}{\glsadd}
% If you are planning to set up your list of symbols manually, then you need to
% add every entry manually.
%    \begin{macrocode}
%<symbols>\glsadd{sym:degrees-of-freedom}
%    \end{macrocode}
% You may likewise use the glossary's definition inside a macro that you use
% later throughout your document. This way, the first appearance of the entry
% will be marked and referenced in the list of glossaries. Your mileage may
% vary.
%
% Following next, we will just add a few more entries to our nomenclature to
% have a nice and long list.
%    \begin{macrocode}
%<*symbols>
\newglossaryentry{sym:number-of-something}{%
  type=symbol,%
  name={\ensuremath{n_{\parentheses{}}}},%
  sort={number of X},%
  unit={},%
  symbol={
    \ensuremath{n_{\parentheses{}}}%
  },%
  description={%
    Number of quantity given in subscript index %
    ${}_{\parentheses{}}$ \eg $\numposconstraints$ %
    is the number of position constraints in a constrained 5
    multi-body system.%
  },%
}
\glsadd{sym:degrees-of-freedom}
%
\newglossaryentry{sym:time}{%
  type=symbol,%
  name={\ensuremath{t}},%
  sort={time},%
  unit={\si{\second}},%
  symbol={
    \ensuremath{t}%
  },%
  description={%
    Time; scalar quantity, described as fundamental quantity %
    and considered an absolute\textemdash not to be confused %
    with absolute time and space\textemdash \ie observed %
    time intervals between pairs of events are the same for %
    all observers.
  },%
}
\glsadd{sym:time}
%
\newglossaryentry{sym:mass-scalar}{%
  type=symbol,%
  name={\ensuremath{\mass_{\ms{B}}}},%
  sort={mass scalar},%
  unit={\si{\kilo\gram}},%
  symbol={
    \ensuremath{\mass_{\ms{B}}}%
  },%
  description={Scalar mass of rigid body~$\mc{B}$},%
}
\glsadd{sym:mass-scalar}
%
\newglossaryentry{sym:mass-matrix}{%
  type=symbol,%
  name={\ensuremath{\Mass_{\ms{B}}}},%
  sort={mass matrix},%
  unit={\si{\kilo\gram}},%
  symbol={
    \ensuremath{\Mass_{\ms{B}}}%
  },%
  description={%
    Multi-dimensional linear mass/inertia of rigid body~$\mc{B}$.%
  },%
}
\glsadd{sym:mass-matrix}
%
\newglossaryentry{sym:identity-matrix}{%
  type=symbol,%
  name={\ensuremath{\eye[c]}},%
  sort={identity matrix},%
  unit={},%
  symbol={
    \ensuremath{\eye[c]}%
  },%
  description={%
    Identity matrix in~$\mathds{R}^{c \times c}$.%
  },%
}
\glsadd{sym:identity-matrix}
%</symbols>
%    \end{macrocode}
% \end{macro}
% </symbols>
%
% \subsubsection{Notation}
%
% Notation is a list similar to regular |glossaries|. As such, you can just follow the lines of the Symbols subsection.
%
% Now, we are able to add all our glossary entries one by one
% \begin{macro}{\newglossaryentry}
%    \begin{macrocode}
%<*notation>
\newglossaryentry{nom:matrix-quantity}{%
  type=notation,%
  name={matrix-quantity},%
  sort=1,%
  unit={},%
  symbol={\ensuremath{\vect{M}}},%
  description={%
    Matrix quantities are, if not introduced otherwise, written in uppercase bold math font.%
  },%
}
%</notation>
%    \end{macrocode}
%
% \end{macro}
%
% \begin{macro}{\glsadd}
% If you are planning to set up your nomenclature manually, then you need to add
% every entry manually.
%    \begin{macrocode}
%<notation>\glsadd{nom:matrix-quantity}
%    \end{macrocode}
% You may likewise use the glossary's definition inside a macro that you use
% later throughout your document. This way, the first appearance of the entry
% will be marked and referenced in the list of glossaries. Your mileage may
% vary.
%
% Following next, we will just add a few more entries to our nomenclature to
% have a nice and long list.
%
%    \begin{macrocode}
%<*notation>
\newglossaryentry{nom:vector-quantity}{%
  type=notation,%
  name={vector-quantity},%
  sort=2,%
  unit={},%
  symbol={\ensuremath{\bm{x}}},%
  description={%
    Vector quantities are, if not introduced otherwise, written %
    in lowercase bold math font.%
  },%
}
\glsadd{nom:vector-quantity}
%
\newglossaryentry{nom:scalar-quantity}{%
  type=notation,%
  name={scalar-quantity},%
  sort=3,%
  symbol={\ensuremath{x}},%
  description={%
    Scalar quantities are, if not introduced otherwise, written %
    in lowercase light math font.%
  },%
}
\glsadd{nom:scalar-quantity}
%
\newglossaryentry{nom:transpose}{%
  type=notation,%
  name={transpose},%
  symbol={\ensuremath{\transpose{\parentheses{\cdot}}}},%
  description={%
    Transpose of vector or matrix~$\parentheses{\cdot}$ such %
    that~$\transpose{\parentheses{\cdot}_{rc}} = \parentheses{\cdot}_{cr}$.%$
  },%
}
\glsadd{nom:transpose}
%
\newglossaryentry{nom:norm}{%
  type=notation,%
  name={norm},%
  symbol={\ensuremath{\norm{\cdot}_{p}}},%
  description={%
  $p$-norm (also called~$\ell_{p}$-norm) of vector~$\vect{v} \in %
  \mathds{R}^{k}$ with~$p \geq 1$ given by~${ \norm{\vect{v}}_{p} = %
  \parentheses[\bigg]{ \sum_{ j = 1}^{ k }{ \pow[p]{ \abs{ \vect{v}_{j} } } } }^{ \nicefrac{1}{p} } }$. %
  If~$p$ is omitted, we assume~$p \equiv 2$.
  },%
}
\glsadd{nom:norm}
%</notation>
%    \end{macrocode}
% \end{macro}
% </symbols>
%
%
% \subsubsection{Acronyms}
% Acronyms are natively supported by the glossaries package, however, need to be defined slightly differently compared to regular glossaries entries.
%
% \begin{macro}{\newacronym}
% Simple acronyms, where the plural form is just the singular form with an
% appended `s' in English or an appended `en' in German, can be defined using
% the |\newacronym| macro. Any acronym is defined with three arguments being |label|, |short|, and |long| form.
%    \begin{macrocode}
%<acronyms>\newacronym{acr:CAD}{CAD}{computer-aided design}
%    \end{macrocode}
% When the plural form is different to the singular form, then we need to take a slightly different route i.e.,
%    \begin{macrocode}
%<*acronyms>
\newacronym[%
  shortplural=DOFs,%
  longplural=degrees of freedom,%
]{acr:DOF}{DOF}{degree of freedom}%
\glsadd{acr:DOF}
%</acronyms>
%    \end{macrocode}
%
% Following next, we will just add a few more entries to our list of acronyms to
% have a nice and long list.
%    \begin{macrocode}
%<*acronyms>
\newacronym{acr:CDPR}{CDPR}{%
  cable-driven parallel robot%
}%
\glsadd{acr:CDPR}
%
\newacronym{acr:NURBS}{NURBS}{%
  non-uniform rational B-spline,%
}%
\glsadd{acr:NURBS}
%
\newacronym{acr:ODE}{ODE}{%
  ordinary differential equation%
}%
\glsadd{acr:ODE}
%
\newacronym{acr:DAE}{DAE}{%
  differential-algebraic equation%
}%
\glsadd{acr:DAE}
%
\newacronym{acr:PDE}{PDE}{%
  partial differential equation%
}%
\glsadd{acr:PDE}
%
\newacronym{acr:IVP}{IVP}{%
  initial value problem%
}%
\glsadd{acr:IVP}
%
\newacronym{acr:BVP}{BVP}{%
  boundary value problem%
}%
\glsadd{acr:BVP}
%</acronyms>
%    \end{macrocode}
% \end{macro}
% 
% \Finale
